%-----------------------------------------------------------------------------
%
%               Template for sigplanconf LaTeX Class
%
% Name:         sigplanconf-template.tex
%
% Purpose:      A template for sigplanconf.cls, which is a LaTeX 2e class
%               file for SIGPLAN conference proceedings.
%
% Guide:        Refer to "Author's Guide to the ACM SIGPLAN Class,"
%               sigplanconf-guide.pdf
%
% Author:       Paul C. Anagnostopoulos
%               Windfall Software
%               978 371-2316
%               paul@windfall.com
%
% Created:      15 February 2005
%
%-----------------------------------------------------------------------------


\documentclass{sigplanconf}[10pt]

% The following \documentclass options may be useful:

% preprint      Remove this option only once the paper is in final form.
% 10pt          To set in 10-point type instead of 9-point.
% 11pt          To set in 11-point type instead of 9-point.
% authoryear    To obtain author/year citation style instead of numeric.
\RequirePackage[english]{babel} % Se fizerem o texto em ingles

\usepackage[utf8]{inputenc}
\usepackage{amsmath}
\usepackage{etoolbox}
\makeatletter
\patchcmd{\maketitle}{\@copyrightspace}{}{}{}
\makeatother


\begin{document}

\special{papersize=8.5in,11in}
\setlength{\pdfpageheight}{\paperheight}
\setlength{\pdfpagewidth}{\paperwidth}

% Uncomment one of the following two, if you are not going for the 
% traditional copyright transfer agreement.

%\exclusivelicense                % ACM gets exclusive license to publish, 
                                  % you retain copyright

%\permissiontopublish             % ACM gets nonexclusive license to publish
                                  % (paid open-access papers, 
                                  % short abstracts)

%\titlebanner{banner above paper title}        % These are ignored unless
%\preprintfooter{short description of paper}   % 'preprint' option specified.

\title{Title Text}
\subtitle{IAJ Group Number: 22}

\authorinfo{Miguel Amaral - 78865, Tiago Vicente - 79620, Francisco Santos - 79719}
           {Instituto Superior Técnico}


\maketitle

\begin{abstract}
This is the text of the abstract.
\end{abstract}

% general terms are not compulsory anymore, 
% you may leave them out
%\terms
%term1, term2

%\keywords
%keyword1, keyword2

\section{Introduction}

Intro

The structure of the report is as follows: Section \ref{decisions} presents the main implementation decisions in the different project topics. Section \ref{optimi} provides information about the optimizations done in the project and their evaluation. Finally, in section \ref{conclusion} we conclude.

\section{Decisions}
\label{decisions}
Lots of text.

\subsection{A*}
nao sei se ha

\subsection{Node Array}

More text.

\subsection{Goal Bounding}
OH YEAH!!!

\subsection{Path Smoothing}

Lots oLots of text.
e text.

Lots of text.

More text.

Lots of text.

More text.

Lots of text.
e text.

Lots of text.

More text.

Lots of text.

More text.

Lots of text.
e text.

Lots of text.

More text.

Lots of text.

More text.

Lots of text.
e text.

Lots of text.

More text.

Lots of text.

More text.

Lots of text.

More text.f text.


\section{Optimizations}
\label{optimi}
More text.
\subsection{Off-line Goal Bound Calculation}
Lots of text.

Lots of text.

More text.

Lots of text.

More text.

Lots of text.

More text.

Lots of text.
\section{Conclusion}
\label{conclusion}
More text.

Lots of text.

More text.

Lots of text.


% We recommend abbrvnat bibliography style.

\bibliographystyle{abbrvnat}

% The bibliography should be embedded for final submission.

\begin{thebibliography}{}
\softraggedright

\bibitem[Smith et~al.(2009)Smith, Jones]{smith02}
P. Q. Smith, and X. Y. Jones. ...reference text...

\end{thebibliography}


\end{document}

%                       Revision History
%                       -------- -------
%  Date         Person  Ver.    Change
%  ----         ------  ----    ------

%  2013.06.29   TU      0.1--4  comments on permission/copyright notices

