%-----------------------------------------------------------------------------
%
%               Template for sigplanconf LaTeX Class
%
% Name:         sigplanconf-template.tex
%
% Purpose:      A template for sigplanconf.cls, which is a LaTeX 2e class
%               file for SIGPLAN conference proceedings.
%
% Guide:        Refer to "Author's Guide to the ACM SIGPLAN Class,"
%               sigplanconf-guide.pdf
%
% Author:       Paul C. Anagnostopoulos
%               Windfall Software
%               978 371-2316
%               paul@windfall.com
%
% Created:      15 February 2005
%
%-----------------------------------------------------------------------------


\documentclass{sigplanconf}[10pt]

% The following \documentclass options may be useful:

% preprint      Remove this option only once the paper is in final form.
% 10pt          To set in 10-point type instead of 9-point.
% 11pt          To set in 11-point type instead of 9-point.
% authoryear    To obtain author/year citation style instead of numeric.
\RequirePackage[english]{babel} % Se fizerem o texto em ingles

\usepackage[utf8]{inputenc}
\usepackage{amsmath}
\usepackage{etoolbox}
\makeatletter
\patchcmd{\maketitle}{\@copyrightspace}{}{}{}
\makeatother


\begin{document}

\special{papersize=8.5in,11in}
\setlength{\pdfpageheight}{\paperheight}
\setlength{\pdfpagewidth}{\paperwidth}

% Uncomment one of the following two, if you are not going for the 
% traditional copyright transfer agreement.

%\exclusivelicense                % ACM gets exclusive license to publish, 
                                  % you retain copyright

%\permissiontopublish             % ACM gets nonexclusive license to publish
                                  % (paid open-access papers, 
                                  % short abstracts)

%\titlebanner{banner above paper title}        % These are ignored unless
%\preprintfooter{short description of paper}   % 'preprint' option specified.

\title{Efficient and Smoothed Pathfinding - Report}
\subtitle{IAJ Group Number: 22}

\authorinfo{Miguel Amaral - 78865, Tiago Vicente - 79620, Francisco Santos - 79719}
           {Instituto Superior Técnico}


\maketitle

\begin{abstract}
In games we often want to find paths from one location to another. In this project, we experimented and evaluated several algorithms used, in the video game industry, to deal with this problem, which is known as Pathfinding.
\end{abstract}

% general terms are not compulsory anymore, 
% you may leave them out
%\terms
%term1, term2

%\keywords
%pathfinding, video games, Artificial Intelligence in Games

\section{Introduction}

Pathfinding is a common problem in video games that consists in finding a good path from a starting point to a goal point avoiding obstacles and minimizing costs. For instance, in a RTS, like Starcraft, if the player has a unit selected and finds an objective that he wants to attack, the game has to calculate the shortest path to it, avoiding possible structures already built.

In our work, we face the problem of finding said path in a dungeon, which in video games is a huge area, typically indoors, that has multiple rooms. So, we implemented and evaluated several techniques used, in the video game industry, to face this dilemma.

The structure of the report is as follows: Section \ref{decisions} presents the main implementation decisions in the different project topics. Section \ref{eval} presents the evaluation of the pathfinding algorithms implemented. Section \ref{optimi} provides information about the optimizations done in the project and their evaluation. Finally, in section \ref{conclusion} we conclude.

\section{Decisions}
\label{decisions}
Lots of text.

\subsection{A*}
nao sei se ha

\subsection{Node Array}

More text.

\subsection{Goal Bounding}
OH YEAH!!!

\subsection{Path Smoothing}

Lots oLots of text.
e text.

Lots of text.

More text.

Lots of text.

More text.

Lots of text.
e text.

Lots of text.

More text.

Lots of text.

More text.

Lots of text.
e text.

Lots of text.

More text.

Lots of text.

More text.

Lots of text.
e text.

Lots of text.
\section{Evaluation}
\label{eval}

Our project was tested using Unity 2017.1.1f1 on Windows 10 version 1709. The specifications of the computer are an Intel Core i7-6700HQ @ 2.60GHz with 16GB DDR4 RAM, a NVIDIA GTX 965M and a NVMe SSD.

Tiago campeao!

More text.

Lots of text.

More text.

Lots of text.

More text.f text.


\section{Optimizations}
\label{optimi}
More text.
\subsection{Off-line Goal Bound Calculation}
Lots of text.

Lots of text.

More text.

Lots of text.

More text.

Lots of text.

More text.

Lots of text.
\section{Conclusion}
\label{conclusion}
More text.

Lots of text.

More text.

Lots of text.


% We recommend abbrvnat bibliography style.

\bibliographystyle{abbrvnat}

% The bibliography should be embedded for final submission.

\begin{thebibliography}{}
\softraggedright

\bibitem[Smith et~al.(2009)Smith, Jones]{smith02}
P. Q. Smith, and X. Y. Jones. ...reference text...

\end{thebibliography}


\end{document}

%                       Revision History
%                       -------- -------
%  Date         Person  Ver.    Change
%  ----         ------  ----    ------

%  2013.06.29   TU      0.1--4  comments on permission/copyright notices

